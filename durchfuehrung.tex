\section{Durchführung}
\label{sec:Durchführung}

Zunächst wird die Probe auf einen festen Abstand eingestellt und  die Kammer mithilfe der Vakuumpumpe evakuiert.
Wenn die Kammer einen Druck von $0\,\text{m}\unit{\bar}$ wird das Messprogramm auf dem Computer gestartet und die Messung beginnt. 
In den 2 Minuten die jede Messung durchgeführt wird, misst das Programm die Häufigkeit der Energien, die die Helium-Kerne besessen haben und hat als Ausgabe werden die Anzahl der 
detektierten $\alpha$-Teilchen, die häufigste Energie derer und die Häufigkeit dieser angezeigt.
Nach der Messung in der evakuierten Kammer wird der Druck in der Kammer auf $50\, \text{m}\unit{\bar}$ erhöht und ein erneuter Messdurchlauf gestartet.
Die Messung wird so lange fortgesetzt bis entweder ein Druck von $p= 1\, \unit{\bar}$ (welcher dem atmosphärischen Druck entspricht) erreicht ist oder keine Teilchen mehr nachgewiesen werden können.
Insgesamt werden so zwei Abstände ausgemessen.
Im Anschluss dazu wird dei Statistik des radioaktiven Zerfalls untersucht.
Hierzu werden Druck, Temperatur und Abstand konstant gehalten und die Messung 100-mal wiederholt.
Gemessen wird 10 $\unit{\second}$ lang.
Es werden jeweils die Anzahl der detektierten Teilchen notiert, um später eine Aussage über die stochastische Verteilung der Häufigkeit von Zerfällen treffen zu können.
