\section{Diskussion}
\label{sec:Diskussion}
Wenn die Messungen der beiden verschiedenen Abstände in Bezug auf die mittlere Reichweite und den Energieverlust verglichen werden, stellt sich heraus, 
dass bei einer geringeren Distanz zwischen der Probe und dem Detektor die mittlere Reichweite sowie die mittlere Energie größer ist. Dies liegt vermutlich
daran, dass durch die geringere Strecke die Alphateilchen durch den herrschenden Druck nicht so stark ausgebremst werden. Entsprechend lässt sich beobachten, 
dass der Energieverlust bei einer größeren Strecke größer ist. Dies stimmt mit der Behauptung des Ausbremsens der Alphateilchen überein.

Da es keine theoretischen Werte als Referenz bzw. Vergleich gibt, bleibt ein Begründen und Abwiegen möglicher Fehler und 
Messunsicherheiten weitestgehend aus.
Einzig und allein kann über die Unsicherheit bei der Geraden zur Bestimmung des Energieverlustes diskutiert werden,
da die Steigung stark davon abhängt, welche Werte man mit in die Berechnung einfließen lässt und welche man auslässt.
Der Übergang verlief fließend, sodass es hier zu Abweichungen kommen kann.
Auch beim Ablesen des Manometers ist ein gewisser Fehler nicht auszuschließen. Weitere Messunsicherheiten sind durch das Verwenden eines 
Computerprogramms auf ein Minimum reduziert worden.

Bei dem Versuch zur Statistik des radioaktiven Zerfalls ist auffällig, dass die Varianz gegenüber den Messwerten sehr groß ist. Auch die Abweichung von der
Poissonverteilung zu der gemessenen Verteilung ist groß. Dies könnte daran liegen, dass nur 100 Messungen durchgeführt wurden. Für eine bessere Verteilung
wären weitaus mehr Messungen nötig gewesen. Außerdem ist es nicht auszuschließen, dass der Druck sich während der Messungen leicht verändert hat, falls das Ventil 
nicht vollständig dicht war.