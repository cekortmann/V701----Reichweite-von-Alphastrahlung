\section{Auswertung}
\label{sec:Auswertung}

\subsection{Fehlerrechnung}
\label{sec:Fehlerrechnung}
Für die Fehlerrechnung werden folgende Formeln aus der Vorlesung verwendet.
für den Mittelwert gilt
\begin{equation}
    \overline{x}=\frac{1}{N}\sum_{i=1}^N x_i ß\; \;\text{mit der Anzahl N und den Messwerten x} 
    \label{eqn:Mittelwert}
\end{equation}
Der Fehler für den Mittelwert lässt sich gemäß
\begin{equation}
    \increment \overline{x}=\frac{1}{\sqrt{N}}\sqrt{\frac{1}{N-1}\sum_{i=1}^N(x_i-\overline{x})^2}
    \label{eqn:FehlerMittelwert}
\end{equation}
berechnen.
Wenn im weiteren Verlauf der Berechnung mit der fehlerhaften Größe gerechnet wird, kann der Fehler der folgenden Größe
mittels Gaußscher Fehlerfortpflanzung berechnet werden. Die Formel hierfür ist
\begin{equation}
    \increment f= \sqrt{\sum_{i=1}^N\left(\frac{\partial f}{\partial x_i}\right)^2\cdot(\increment x_i)^2}.
    \label{eqn:GaussMittelwert}
\end{equation}

\subsection{Reichweite von Alphastrahlung}

Um die Reichweite von Alphastrahlung zu bestimmen muss der gemessene Channel des jeweiligen Drucks zuerst in MeV umgerechnet werden. Dabei entspricht der
Channel bei einem Druck $p = 0\,\symup{mbar}$ einer Energie $E = 4 \,\unit{\MeV}$. Diese fällt dann linear ab. Neben dem Druck, der Zählrate, dem 
Channel, der Energie und der Zählrate des jeweiligen Energiemaximums ist in \autoref{tab:3cm} auch noch die effektive Länge $x$ eingetragen. Diese
berechnet sich durch \autoref{eqn:effLaeng}. Die Messung wurde bei einem Abstand zwischen Detektor und Probe von $x_0 = 3\,\unit{\cm}$ durchgeführt.

\begin{table}
    \centering
    \caption{Druckabhängigkeit der Zählrate, der Energie und der effektiven Länge bei einem Abstand zwischen Probe und Detektor von $x_0 = 3\,\unit{\cm}$.}
\begin{tabular}{c c c c c}
    \toprule
    $p \mathbin{/}\symup{mbar}$ &Zählrate& Channel & $E \mathbin{/}\unit{\MeV}$ & Zählrate des Maximums & $x \mathbin{/}\unit{\m}$ \\
    \midrule
    0&55341&831&4.0000&190&0.0000
    50&54428&816&3.9278&195&0.0015
    100&53962&791&3.8075&201&0.0030
    150&53946&775&3.7304&213&0.0044
    200&52866&715&3.4416&234&0.0059
    250&52531&663&3.1913&234&0.0074
    300&51795&615&2.9603&242&0.0089
    350&51292&611&2.9410&263&0.0104
    400&50646&583&2.8063&252&0.0118
    450&49603&547&2.6330&256&0.0133
    500&48743&527&2.5367&276&0.0148
    550&47524&492&2.3682&261&0.0163
    600&45813&431&2.0746&268&0.0178
    650&42665&371&1.7858&278&0.0192
    700&36549&348&1.6751&283&0.0207
    750&24399&271&1.3045&288&0.0222
    800&15029&262&1.2611&256&0.0237
    850&3142&97&0.4669&262&0.0252
    900&92&6&0.0289&262&0.0267
    950&10&1&0.0048&274&0.0281
    1000&1&1&0.0048&259&0.0296
    \bottomrule
    \end{tabular}
    \label{tab:3cm}
\end{table}

Die letzten Messungen der Energie sind aufgrund der geringen Zählrate statistisch nicht mehr aussagekräftig.






Bei einer zweiten Messung wurde der Abstand zwischen Detektor und Probe auf $x_0 = 4.5 \,\unit{\cm}$ variiert. In \autoref{tab:4.5cm} sind erneut die 
Messdaten zu Zählrate, Channel, Energie, Zählrate des Energiemaximums sowie der effektiven Länge dargestellt.

\begin{table}
  \centering
  \caption{Druckabhängigkeit der Zählrate, der Energie und der effektiven Länge bei einem Abstand zwischen Probe und Detektor von $x_0 = 4.5\,\unit{\cm}$.}
\begin{tabular}{c c c c c}
  \toprule
  $p \mathbin{/}\symup{mbar}$ &Zählrate& Channel & $E \mathbin{/}\unit{\MeV}$ & Zählrate des Maximums & $x \mathbin{/}\unit{\m}$ \\
  \midrule
                    0&31206&830&4.0000&109&0.0000 \\
                    50&30522&768&3.7012&125&0.0022 \\
                   100&30102&745&3.5904&121&0.0044 \\
                   150&29734&675&3.2530&137&0.0067 \\
                   200&29168&668&3.2193&146&0.0089 \\
                   250&28700&631&3.0410&147&0.0111 \\
                   300&28323&559&2.6940&159&0.0133 \\
                   350&27222&536&2.5831&155&0.0155 \\
                   400&26401&463&2.2313&162&0.0178 \\
                   450&25624&412&1.9855&162&0.0200 \\
                   500&22454&367&1.7687&168&0.0222 \\
                   550&16202&295&1.4217&162&0.0244 \\
                     600&3458&271&1.3060&76&0.0267 \\
                       650&107&286&1.3783&6&0.0289 \\
                         700&7&462&2.2265&1&0.0311 \\
                         750&1&315&1.5181&1&0.0333 \\
                           800&0&0&0.0000&0&0.0355 \\
  \bottomrule
  \end{tabular}
  \label{tab:4.5cm}
\end{table}






\subsection{Statistik des radioaktiven Zerfalls}
In \autoref{fig:histo} befindet sich das Histogramm zur gemessenen Statistik des radioaktiven Zerfalls.

\begin{figure}
  \centering
  \includegraphics[height = 6cm]{build/histo.pdf}
  \caption{Histogramm zur Statistik des radioaktiven Zerfalls.}
  \label{fig:histo}
\end{figure}

Der Mittelwert berechnet sich mittels \autoref{eqn: Mittelwert} zu $\bar{N} = 2443.23 \pm 9.49$. Die Varianz
der Messung liegt bei ${\Delta\overline{N}}^2 = 8913.197$.