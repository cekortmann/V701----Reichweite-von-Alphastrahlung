\section{Vorbereitungsaufgaben}
\label{sec:vorbereitung}

Ein Halbleiterzähler besteht aus je einem Halbleiter mit n-Leitung und p-Leitung. An der Kontaktstelle entsteht durch Diffusion eine Zone mit
beweglichen Ladungsträgern. Dies passiert solange bis das so aufgebaute elektrische Feld die weitere Diffusion verhindert. Der Übergang zwischen 
den Halbleitern ähnelt einer Diode. Wenn nun der n-Bereich mit einer Anode und der p-Bereich mit einer Kathode verbunden wird, vergrößert sich
der Bereich der freien Ladungsträger. Dieser Bereich wird auch Sperrzone genannt.

Wenn ein ionisierendes Teilchen durch diese Zone durchdringt, werden Elektronen und Löcher erzeugt. dadurch entsteht ein kurzzeitiger Stromfluss,
der messbar ist.