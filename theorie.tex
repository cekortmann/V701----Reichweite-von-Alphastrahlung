\section{Theorie}
\label{sec:Theorie}

Durch die Messung der Reichweite von Alphastrahlung kann auch dessen
Energie bestimmt werden. Durch Anregung und Dissoziation von Molekülen beim 
Durchlaufen eines Materials verliert die Alphastrahlung an Energie. Dieser
Energieverlust ist abhängig von der Energie der Alphastrahlung sowie der
Dichte des durchlaufenen Materials und berechnet sich über die Bethe-Bloch-Gleichung
\begin{equation}
    -\frac{d E_\alpha}{d x}=\frac{z^2 e^4}{4 \pi \epsilon_0 m_e} \frac{n Z}{v^2} \ln \left(\frac{2 m_e v^2}{I}\right) \; .
    \label{eqn: energieverlust}
\end{equation}
Dabei ist $z$ die Ladung und $v$ die Geschwindigkeit der Alphastrahlung; $Z$ die Ordnungszahl,
$n$ die Teilchendichte und $I$ die Ionisierungsenergie des Targetgases. 

Die Reichweite $R$ ist die Strecke bis zur vollständigen Abbremsung eines 
Alphateilchens und berechnet sich durch 
\begin{equation*}
    R=\int_0^{E_\alpha} \frac{d E_\alpha}{-d E_\alpha / d x} \; .
\end{equation*}
Bei kleinen Energien nehmen Ladungsaustauschprozesse stark zu und der
Energieverlust lässt sich nicht mehr durch die Bethe-Bloch-Gleichung 
\eqref{eqn: energieverlust} beschreiben. Stattdessen nutzt man zur Bestimmung
der mittleren Reichweite $R_{\symup{m}}$ empirisch bestimmte Kurven. Für 
Energien $E_{\symup{\alpha}} \leq 2.5 \, \unit{\MeV}$ lässt sich die mittlere 
Reichweite über $R_{\symup{m}} = 3.1 \cdot E_{\symup{\alpha}}^{3/2}$ bestimmen.

In Gasen ist die Reichweite der Alphastrahlung proportional zum Druck und kann
über eine Absorptionsmessung bestimmt werden, bei der der Druck variiert wird.
Für die Reichweite gilt dann 
\begin{equation*}
    x = x_0 \frac{p}{p_0} \, .
\end{equation*}
Dabei ist $x_0$ der Abstand zwischen dem Detektor und dem Alphastrahler. 
$p_0$ ist der Normaldruck $p_0 = 1013 \, \symup{mbar}$.